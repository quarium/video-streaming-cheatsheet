%------------------------------------------------
% Tech
%------------------------------------------------
\section{Technical}
\hrule

\medskip

\textbf{3ivX}
\textit{MPEG-4} compliant video codec suite.

%------------------------------------------------
\subsection{A}

\smallskip
\textbf{AAC (Advanced Audio Coding)}
Audio codec for online video and live streaming.

\smallskip
\textbf{Adaptive bitrate streaming}
detecting a user's bandwidth and CPU capacity in real time and adjusting the quality of the media stream accordingly. [2] It requires the use of an encoder which can encode a single source media (video or audio) at multiple bit rates.

\smallskip
\textbf{AIFF (Audio Interchange File Format)}
Audio format developed by Apple.

\smallskip
\textbf{ASX}
A metafile which points to a Windows Media audio/video presentation. See also ASF.

\smallskip
\textbf{ASF (Advanced System Format)}
A format defined by Microsoft for media files.

\smallskip
\textbf{AVI (Audio Video Interleave)}
Proprietary \textit{Container} format by Microsoft, can contain both audio and video data in a file container.

\smallskip
\textbf{AVC (Advanced Video Coding)}
See \textit{H.264}.

%------------------------------------------------
\subsection{B}

\smallskip
\textbf{B Frame (Bidirectional predictive picture frame)}
Bidirectional predictive frames or pictures (B frames) reference a P or predicitve frame and then predict movement across multiple B frames and between two P frames. In interframe compression, the B (bidirectional) frame references the P frame, which in turns references the I frame or keyframe.


%------------------------------------------------
\subsection{C}

\smallskip
\textbf{CBR (Constant Bitrate)}
Encoding each frame or group of pictures (GOP) at a constant bitrate, under the assumption that transmission rates will remain constant.

\smallskip
\textbf{CDN}
Content Delivery Network

\smallskip
\textbf{CENC (Common Encryption Scheme)}
Common Encryption Scheme is a standards-based approach to provide encryption and key-mapping methods for any digital rights management (DRM) system used with the ISO Base Media File Format (see ISO BMFF). Rather than providing the rights management itself, CENC defines encryption-related metadata in a standard format, allowing for consistent decryption of protected streams.

\smallskip
\textbf{Chrominance}
The color of a video signal. See also \textit{Y (Luminance)}.

\smallskip
\textbf{Chunking}
See "Fragmentation".

\smallskip
\textbf{Codec}
Compress \& decompress (encode \& decode) standard for reducing the file sizes of video \& audio. Most common is \textit{H.264}, others are \textit{MPEG-2}, \textit{HEVC}, \textit{VP9}, \textit{Quicktime} and \textit{WMV}.

\smallskip
\textbf{Container}
Holds together the audio and video codecs and the metadata for the video that is being encoded. The container stores all of this information in one file.

\smallskip
\textbf{CTA}
Lorem

%------------------------------------------------
\subsection{D}

\smallskip
\textbf{dash.js}
dash.js is a reference client implementation by the DASH Industry Forum (DASH-IF) for the playback of MPEG-DASH via JavaScript and compliant MSE/EME platforms.

\smallskip
\textbf{DAM (Digital Asset Managment)}
See "MAM".

\smallskip
\textbf{DASH (Dynamic Adaptive Streaming via HTTP)}
Adaptive bitrate streaming technique that enables high quality streaming of media content over the Internet delivered from conventional HTTP web servers. Works by breaking the content into a sequence of small segments, which are served over HTTP. See also \textit{HLS}.

\smallskip
\textbf{DRM (Digital Rights Management)}
Access control technologies that limit the usage of digital content.

\smallskip
\textbf{DOCSIS (Data Over Cable Service Interface Specification)}
International telecommunications standard that permits the addition of high-bandwidth data transfer to an existing cable television (CATV) system

\smallskip
\textbf{Deinterlace}
To process interlaced television video, deinterlacing uses every other line from one field and interpolates new in-between lines. See also \textit{Interlace}.

\smallskip
\textbf{DVB}
Lorem

%------------------------------------------------
\subsection{E}

\smallskip
\textbf{Encode}
Either the initial process of compressing RAW video or to the process of re-encoding a video into a different format.

\smallskip
\textbf{Encrypted Media Extensions}
An API that enables web applications to interact with content protection systems, to allow playback of encrypted audio and video. Designed to enable the same app and encrypted files to be used in any browser, regardless of the underlying protection system.

%------------------------------------------------
\subsection{F}

\smallskip
\textbf{FairPlay}
Apple DRM - Cipher Block Chaining encryption, the only option for Safari and is only used by Apple devices.

\smallskip
\textbf{Fragmentation}
Process of dividing a larger file into a series of smaller files with the purpose of delivering them over the internet.

\textbf{Frame Rate}
The number of frames per second (FPS) for cinema or video acquisition, editing and projection. Common frame rates include 24 fps (film), 30 fps, and 29.97 fps.

\textbf{FPS (Frames Per Second)}
See \textit{Frame Rate}.

\smallskip
\textbf{Fingerprint}
Lorem. See also \textit{Watermark}.

\smallskip
\textbf{fMP4 (Fragmented MP4)}
Contains a series of segments which can be requested individually if the server supports byte-range requests.

\smallskip
\textbf{F4V (Flash H.264 Video)}
Flash MP4 Video, an Adobe proprietary extension used to designate MPEG-4 H.264 content in a Flash-compliant container format. IThe F4V file format can contain data corresponding to the ActionScript Message Format and still frame of video data using image formats GIF, JPEG and PNG.

%------------------------------------------------
\subsection{G-H}

\smallskip
\textbf{HDTV (High Definition Television)}
Broadcast and viewing of high-definition signals (720p progressive or 1080i).

\smallskip
\textbf{H.261}
A standards-based video compression technique used primarily for videoconferencing systems, given its low-latency acceptable performance.

\smallskip
\textbf{H.262 (MPEG-2)}
More commonly called the MPEG-2 Video codec

\smallskip
\textbf{H.263}
A standards-based video compression technique used primarily for videoconferencing systems.

\smallskip
\textbf{H.264}
\textit{Codec}, also known as \textit{AVC} \&  \textit{MPEG-4 Part 10}.

\smallskip
\textbf{H.265}
H.265 is a codec draft specification (currently at Working Draft 6) that is set to supersede H.264.


\smallskip
\textbf{HDS (HTTP Dynamic Streaming)}
\textit{Adaptive bitrate streaming} method for on-demand and live developed by Adobe. MP4 video content over HTTP connections.

\smallskip
\textbf{HEVC (High Efficiency Video Coding)}
See H.265

\smallskip
\textbf{HbbTV (Hybrid Broadcast Broadband TV)}
Standard for delivering content to consumers for consumption on TV displays over broadcast and broadband networks. It uses elements of existing solutions including ones from \textit{W3C}, \textit{MPEG}, \textit{DVB}, \textit{CTA} \& \textit{OIPF}.

\smallskip
\textbf{HLS (HTTP Live Streaming)}
Protocol for delivering live streams over the internet.

\smallskip
\textbf{GOP}
Group of Pictures, a grouping of frames (pictures) for interframe compression techniques. Longer GOPs allow for better multi-frame compression but slow down the overall delivery process for live encoding. Each GOP begins with an I-frame keyframe and then can contain any number of predictive (P frames) and bidirectional predictive (B frames).


%------------------------------------------------
\subsection{I-K}

\smallskip
\textbf{Interlace}
Each full frame of video consists of alternating lines taken from two separate fields captured at slightly different times. The two fields are then interlaced or interleaved into the alternating odd and even lines of the full video frame.  See also \textit{Deinterlace}.

\smallskip
\textbf{ISO BMFF (ISO Base Media File Format)}
Lorem ipsum dolor sit amet, quem facer mentitum per ei. At omnes appetere eos, sea ut noster perfecto.

\smallskip
\textbf{I Frame (Intra picture frame)}
Lorem ipsum dolor sit amet, quem facer mentitum per ei. At omnes appetere eos, sea ut noster perfecto.

\smallskip
\textbf{Keyframe}
In video encoding, a keyframe stores the complete image of the scene.



%------------------------------------------------
\subsection{L}

\smallskip
\textbf{LightningJS}
Lightning is a fast WebGL Application Framework makes use of hardware accelerated rendering that creates high quality animations on every device.

\smallskip
\textbf{Luminance (Y)}
The intensity or brightness of a video signal. See also \textit{Chrominance}.

\smallskip
\textbf{LTC (Longitudinal Timecode)}
\textit{Timecode} that is added outside the vertical interval, so that video tape recorders used for editing can maintain a reference point to timecode even when the VTR is in fast-forward or jog-shuttle modes.

\smallskip
\textbf{LITC (Linear Timecode)}
\textit{SMPTE} \textit{Timecode} data in an audio signal.

%------------------------------------------------
\subsection{M}

\smallskip
\textbf{MSE}
Media Source Extensions

\smallskip
\textbf{m3u8}
m3u8 is a standardized type of playlist file. In the online video industry, m3u8 files can be used to define a playlist for delivery to smart TV devices such as Roku.

\smallskip
\textbf{Multi-bitrate streaming}
sending out multiple versions of your live stream feed at different quality levels

\smallskip
\textbf{MP4}
Common container format

\smallskip
\textbf{MPEG-DASH}
See \textit{DASH}

\smallskip
\textbf{MPEG-1}
Video standard for compression of audio and video signals into a disk-based format. MPEG-1 attempted to match VHS in quality.

\smallskip
\textbf{MPEG-2}
Lorem ipsum dolor sit amet, quem facer mentitum per ei. At omnes appetere eos, sea ut noster perfecto.

\smallskip
\textbf{MPEG-4}
Lorem ipsum dolor sit amet, quem facer mentitum per ei. At omnes appetere eos, sea ut noster perfecto.

\smallskip
\textbf{MPEG-H}
Lorem ipsum dolor sit amet, quem facer mentitum per ei. At omnes appetere eos, sea ut noster perfecto.

\smallskip
\textbf{Multicast}
Lorem ipsum dolor sit amet, quem facer mentitum per ei. At omnes appetere eos, sea ut noster perfecto.

\smallskip
\textbf{MAM (Media Asset Management)}
Media Asset Management is a specialized form of a content management system (CMS). The MAM is a repository for large media files, such as video and full-resolution images. MAM systems, software, and platforms are often interchangeably called DAMs.

\smallskip
\textbf{MPEG-4 Part 10}
See \textit{H.264}.

\smallskip
\textbf{MP3}
Popular audio compression container and codec, using the MP3 extension.

\smallskip
\textbf{Marlin}
Open-standard DRM. Most widely used DRMs in Japan, China, and parts of East Asia. Panasonic, Philips, Samsung, and Sony. Together, they created a DRM standard that would not only be used for their own devices, but also could be adopted globally


\smallskip
\textbf{MP4}
Container Format.

\smallskip
\textbf{Multiplex}
A term describing the interleaving of multple signals together into a single signal. Multiplexing is often shortened even further to the term "muxing" and a pertient example of muxing is the conversion of MPEG-2 program or elementary streams into an MPEG-2 transport stream for broadcast delivery

\smallskip
\textbf{Muxing}
See "Multiplex".


%------------------------------------------------
\subsection{N}

\smallskip
\textbf{NDI}
NDI

\smallskip
\textbf{Narrowcast}
To send data to a specific list of recipients.

\smallskip
\textbf{NTSC}
A television video format used in the United States and elsewhere. Displayed 525 lines of resolution at 60 fields per second, 30 frames per second (actually a fractional value near 29.97). Named for the National Television Standards Committee. See also \textit{PAL}.


%------------------------------------------------
\subsection{O}

\smallskip
\textbf{Ogg}
An open-source container format that primarily houses \textit{Theora} and \textit{Vorbis}.

\smallskip
\textbf{OpenGL}
It is written in C language, It is a cross-language and platform API to render 2D and 3D vector graphics normally used for desktop Apps.

\smallskip
\textbf{OBS}
Openbroadcaster

%------------------------------------------------
\subsection{P}

\smallskip
\textbf{Progressive Download}
A method of delivering audio/video data over the Internet that involves playing a segment, often the initial downloaded portion of a file, while the remaining content download is still in progress.

\smallskip
\textbf{Progressive Video}
Video consisting of complete frames, not interlaced fields. Each individual frame is a coherent image captured by the camera at a single moment in time. See also interlaced video.

\smallskip
\textbf{PlayReady}
Microsoft DRM - Supported on Windows, most set-top boxes and TVs, uses WRMHEADER tag objects as metadata format

\smallskip
\textbf{Perceptual Compression}
A compression technique that takes advantage of knowledge of how humans perceive; that is, by eliminating visual detail that the eye cannot easily see or audio frequencies that the ear cannot easily hear.

\smallskip
\textbf{PAL (Phase Alternation Line)}
Television video format used in Europe and elsewhere. Displayed with 625 lines of resolution at 50 fields per second, 25 frames per second. See also \textit{NTSC}.


%------------------------------------------------
\subsection{Q-R}

\smallskip
\textbf{RTSPE}
RTSPE

\smallskip
\textbf{RTSP}
RTSP

\smallskip
\textbf{RTMP (Real-Time Multimedia Protocol)}
Protocol for live streaming often use to deliver a stream from encoder to OVP endpoint for ingestion.

\smallskip
\textbf{RTP (Real-time Transport Protocol)}
Network protocol for delivering audio and video over IP networks.

\smallskip
\textbf{RGB}
Acronym for Red, Green, Blue. Full-color video signal format, consisting of three elements. See also YUV.


%------------------------------------------------
\subsection{S}

\smallskip
\textbf{Segmenting}
See "Fragmentation".

\smallskip
\textbf{Simulcast (Simultaneous Broadcast)}
A broadcast that is coordinated between two or more mediums, such as radio and television.

\smallskip
\textbf{Smooth Streaming}
Microsoft’s proprietary adaptive-bitrate (ABR) technology, Smooth Streaming uses fragmented MP4 (ISO Base Media File Format) and the Protected Interoperable File Format (PIFF) to deliver small-file-segment streaming via HTTP. Smooth Streaming has been somewhat deprecated by Microsoft’s embrace of DASH as an industry standard.

\smallskip
\textbf{SVC (Scalable Video Coding)}
Lorem

%------------------------------------------------
\subsection{T}

\smallskip
\textbf{TS (Video Transport Stream)}
Video data compressed with standard MPEG-2. Often utilised for saving video on a DVD or broadcast video.


\smallskip
\textbf{TriCaster}
Enterprise encoders from made by NewTek.

\smallskip
\textbf{TeraDek}
TeraDek is a producer of high-quality hardware encoders.


\smallskip
\textbf{Theora}
Open-source video \textit{codec}..

\smallskip
\textbf{Transcode}
Process of creating copies of video files in different sizes.

\smallskip
\textbf{Timecode}
An exact time used to identify a specific frame in a clip or production. Measured in hours, minutes, seconds, and frames. Primary types are \textit{SMPTE}, \textit{LITC} and \textit{VITC}.

\smallskip
\textbf{Timecode}
A measurement of video frames, using an absolute reference point between each field or frame of video. Two primary types of

%------------------------------------------------
\subsection{U}

\smallskip
\textbf{UltraViolet}
Defunct cloud-based digital rights locker for films and television programs that allowed consumers to store proofs-of-purchase of licensed content in an account to enable playback on different devices using multiple applications from several different streaming services.

\smallskip
\textbf{Ultra HD}
The industry brand name for consumer 4K monitors is Ultra HD (UHD)

%------------------------------------------------
\subsection{V}

\smallskip
\textbf{VP9}
VP9 is an open and royalty-free video coding format developed by Google, competes mainly with MPEG's High Efficiency Video Coding (HEVC/H.265).

\smallskip
\textbf{VBR (Variable Bitrate)}
VBR allows a higher bitrate (and therefore more storage space) to be allocated to the more complex segments of media files while less space is allocated to less complex segments.

\smallskip
\textbf{Vorbis}
Open-source audio \textit{codec}.

\smallskip
\textbf{VidBlasterX}
Software encoder for producing and streaming live events.

\smallskip
\textbf{vMix}
Software encoder for producing and streaming live events.

\smallskip
\textbf{VLC (VideoLan Client)}
A player capable of playing a wide variety of formats.

\smallskip
\textbf{VITC (Vertical Interval Timecode)}
\textit{SMPTE} \textit{Timecode} encoded on one scan line in a video signal.

\smallskip
\textbf{VBR}
Acronym for Variable Bit Rate. A compression scheme in which each unit of input material can be compressed to different sizes. For MPEG-2 video, for example, this means that "easier" sequences (that is, with no motion) can compress to very small sizes, whereas "hard" sequences (with lots of motion and scene cuts) can compress to much larger sizes. VBR compression can take better advantage of the overall available bandwidth of a video transmission or DVD player by allocating the available bits intelligently to the difficult parts of a sequence. See also CBR.


%------------------------------------------------
\subsection{W}

\smallskip
\textbf{WebRTC (Web Real-Time Communication)}
Provides web browsers and mobile applications with real-time communication via APIs. Allows direct peer-to-peer communication, eliminating the need to install plugins.

\smallskip
\textbf{WMVHD (Windows Media Video High Definition)}
Marketing name for high definition videos encoded using Microsoft Windows Media Video 9 codecs.

\smallskip
\textbf{Windows Media}
Microsoft's proprietary legacy media format that housed audio, video, and metadata.

\smallskip
\textbf{WMV (Windows Media Video)}
Windows Media Video is a proprietary video format, housing Windows Media 9 and 9 Advanced codecs, the latter of which became the VC-1 standard.

\smallskip
\textbf{WebM}
A video compression codec created by On2 Technologies (as VP8), purchased by Google and then open-sourced

\smallskip
\textbf{Widevine}
Google DRM - Used on Android Devices natively, in Chrome, Edge (soon), Roku, Smart TVs, uses protobuf format for metadata.

\smallskip
\textbf{Watermark}
Lorem. See also \textit{Fingerprint}.

\smallskip
\textbf{WebGL}
Designed for rendering 2D and 3D graphhics, running in the browser with Javascript. See also \textit{OpenGL}.

\smallskip
\textbf{WMV (Windows Media Video)}
Series of video codecs and their corresponding video coding formats developed by Microsoft.

\smallskip
\textbf{Wirecast}
Software encoder for producing and streaming live events, made by Telestream.

%------------------------------------------------
\subsection{X-Z}

\smallskip
\textbf{YUV}
Full-color video signal format, consisting of three elements: Y (luminance), and U and V (chrominance). See also RGB.
