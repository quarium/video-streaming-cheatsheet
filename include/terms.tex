%------------------------------------------------
% Terms
%------------------------------------------------

%------------------------------------------------
\section{A}
\hrule

\medskip
\textbf{AAC (Advanced Audio Coding)}
Audio \textit{Codec}.

\smallskip
\textbf{ABR (Adaptive Bitrate Streaming)}
Client selects an appropriate \textit{Encode} quality dynamically by analysing the available bandwidth and CPU.

\smallskip
\textbf{ADI (Asset Distribution Interface Specification)}
Means by which content and metadata are transported from a provider to a \textit{MAM}.

\smallskip
\textbf{AIFF (Audio Interchange File Format)}
Audio format developed by Apple.

\smallskip
\textbf{ARPU (Average Revenue Per User)}
Average revenue generated by each user of service.

\smallskip
\textbf{ASF (Advanced System Format)}
A \textit{Container} format defined by Microsoft for media files.

\smallskip
\textbf{ATSC (Advanced Television Systems Committee)}
Not-for-profit, developing voluntary standards for digital television.

\smallskip
\textbf{AV1 (AOMedia Video 1)}
Open, royalty-free \textit{Codec}. Competes with \textit{H.265}.

\smallskip
\textbf{AVI (Audio Video Interleave)}
Proprietary \textit{Container} format by Microsoft.

\smallskip
\textbf{AVC (Advanced Video Coding)}
See \textit{H.264}.

\smallskip
\textbf{AVOD (Advertising VOD)}
Monetisation Model - Advertising supporting \textit{VOD}.


%------------------------------------------------
\section{B-C}
\hrule

\medskip
\textbf{Cablelabs (Cable Television Laboratories)}
Not-for-profit innovation and R\&D lab.

\smallskip
\textbf{CBR (Constant Bitrate)}
Encoding each frame or \textit{GOP} at a constant bitrate.

\smallskip
\textbf{CC (Closed Captions)}
Text version of the spoken part of a television, movie.

\smallskip
\textbf{CDN (Content Delivery Network)}
Network allowing content to be physically close to the end-user reducing the latency incurred in retrieving content.

\smallskip
\textbf{CENC (Common Encryption Scheme)}
Standard format for encryption-related metadata for any \textit{DRM}. Videos can be encrypted once and any decryption module can decrypt.  Used with \textit{ISO BMFF}.

\smallskip
\textbf{CMAF (Common Media Application Format)}
\textit{Container} and standards for a single approach to streaming that works with \textit{HLS} and \textit{DASH}.

\smallskip
\textbf{CPM (Cost Per Mile)}
Metric that represents how much money advertisers are spending to show ads.

\smallskip
\textbf{Chrominance}
The colour of a video signal. See also \textit{Luminance (Y)}.

\smallskip
\textbf{Chunking}
See \textit{Fragmentation}.

\smallskip
\textbf{Codec (Compress \& Decompress)}
Technique for reducing the file sizes of video \& audio. e.g. \textit{H.264}.

\smallskip
\textbf{Container}
Wraps into a single file the encoded Audio, Video \& Metadata.  e.g. \textit{MP4} \& \textit{AVI}.


%------------------------------------------------
\section{D}
\hrule

\medskip
\textbf{D2C (Direct-to-Consumer)}
Business model to deliver content to customers.

\smallskip
\textbf{DAI (Dynamic Ad Insertion)}
Server-side technology that allows the serving of video ads into live linear programming and \textit{VOD} in a seamless TV-like experience without latency or buffering between content and ads.

\smallskip
\textbf{DAM (Digital Asset Managment)}
See \textit{MAM}.

\smallskip
\textbf{DASH (Dynamic Adaptive Streaming via HTTP)}
\textit{ABR} enabling streaming of content via HTTP server. Doesn't require a specific \textit{codec} - could be \textit{H.264}, \textit{VP9}, \textit{H.265}, etc.

\smallskip
\textbf{DASH-IF (DASH Industry Forum)}
Industry group creating interoperability guidelines and promoting \textit{MPEG-DASH}.

\smallskip
\textbf{dash.js}
Reference client implementation by \textit{DASH-IF} for the playback of \textit{MPEG-DASH} via JavaScript and compliant \textit{MSE} / \textit{EME} platforms.

\smallskip
\textbf{DECE (Digital Entertainment Content Ecosystem)}
Inventor of \textit{Ultraviolet}, now dissolved.

\smallskip
\textbf{Deinterlace}
To process interlaced TV video. Uses every other line from one field \& adds new in-between lines. See also \textit{Interlace}.

\smallskip
\textbf{DMCA (Digital Millennium Copyright Act)}
Law that criminalises production and dissemination of technology, devices, or services intended to circumvent measures that control access to copyrighted works.

\smallskip
\textbf{DOCSIS (Data Over Cable Service Interface Specification)}
Telco standard for adding high-bandwidth data transfer to an existing cable television system.

\smallskip
\textbf{DOG (Digital on-screen graphic)}
\textit{Watermark} logo that most television broadcasters overlay over a portion of the screen area of their programs to identify the channel.

\smallskip
\textbf{DRM (Digital Rights Management)}
Technologies that limit the usage \& sharing of digital content.  See \textit{Widevine}, \textit{Playready}, \textit{Fairplay} \& \textit{Marlin}.

\smallskip
\textbf{DVB (Digital Video Broadcasting)}
Open standards for digital TV.  Satalite [DVB-S, DVB-S2], Cable [DVB-SH, DVB-C, DVB-C2], Terrestrial TV [DVB-T, DVB-T2].


%------------------------------------------------
\section{E-F}
\hrule

\medskip
\textbf{EME (Encrypted Media Extensions)}
API that enables web applications to interact with content protection systems. Enables encrypted files to be used in any browser regardless of the underlying protection system. Only capable of decrypting media data provided via \textit{MSE}.

\smallskip
\textbf{Encode}
Process of compressing RAW video, or the process of re-encoding a video into a different format.

\smallskip
\textbf{F4V (Flash H.264 Video)}
Adobe proprietary extension to H.264 \textit{Container}.

\smallskip
\textbf{FairPlay}
Apple \textit{DRM}, the only option for Safari and is only used by Apple devices.

\smallskip
\textbf{FFMPEG}
Open-source suite of libraries and programs for handling video \& audio.

\smallskip
\textbf{Fingerprint (Perceptual Hash)}
A generated hash of a video that will uniqely identify the video. Not to be confused with \textit{Watermark}.

\smallskip
\textbf{fMP4 (Fragmented MP4)}
Single \textit{MP4} file which contains a series of internal segments which can be requested individually from a HTTP server that supports byte-range requests.

\smallskip
\textbf{FPS (Frames Per Second)}
See \textit{Frame Rate}.

\smallskip
\textbf{Fragmentation}
Divide large media files into a series of smaller files with the purpose of delivering them over the internet.

\smallskip
\textbf{Frame Rate}
The number of frames per second (FPS). Common frame rates include 24 fps (film), 30 fps, and 29.97 fps.


%------------------------------------------------
\section{G-H}
\hrule

\medskip
\textbf{GOP (Group of Pictures)}
Logical grouping of \textit{frames} for compression pruposes. Each GOP begins with an I-frame keyframe and can contain any number of predictive (P frames) and bidirectional predictive (B frames).

\smallskip
\textbf{H.26[1|2|3|4|5|6]}
Standards-based Codecs:\\
 - \textit{H.262} also known as \textit{MPEG-2}.\\
 - \textit{H.264} also known as \textit{AVC} or  \textit{MPEG-4 Part 10}.\\
 - \textit{H.265} also known as \textit{HEVC} or \textit{MPEG-H}. 25\% to 50\% better compression at the same level of video quality, or substantially improved video quality at the same bit rate. It supports resolutions up to 8192×4320, including 8K \textit{UHD}.\\
 - \textit{H.266} also known as \textit{VVC}.

\smallskip
\textbf{HbbTV (Hybrid Broadcast Broadband TV)}
Standard for delivering content to consumers over broadcast and broadband networks.

\smallskip
\textbf{HDS (HTTP Dynamic Streaming)}
Adobe's \textit{ABR} method for on-demand and live - \textit{MP4} video content over HTTP.

\smallskip
\textbf{HDTV (High Definition Television)}
Broadcast and viewing of high-definition signals (720p progressive or 1080i).

\smallskip
\textbf{HEVC (High Efficiency Video Coding)}
See \textit{H.265}.

\smallskip
\textbf{HLS (HTTP Live Streaming)}
Protocol for delivering live streams by HTTP.  \textit{H.264} has to be used as the video \textit{codec}.


%------------------------------------------------
\section{I-K}
\hrule

\medskip
\textbf{I Frame (Intra picture frame)}
See \textit{GOP}.

\smallskip
\textbf{IETF (Internet Engineering Task Force)}
Responsible for the technical standards of TCP/IP.

\smallskip
\textbf{Interlace}
Technique for doubling the perceived frame rate of a video display without consuming extra bandwidth. See also \textit{Deinterlace}.

\smallskip
\textbf{Interstitial}
Any web-based advertisement which occurs before, after, or during the main content.

\smallskip
\textbf{IPTV (Internet Protocol television)}
TV content over IP as opposed to terrestrial, satellite, or cable.

\smallskip
\textbf{ISO BMFF (ISO Base Media File Format)}
\textit{Container} file format.  Basis for other media file formats (e.g. the \textit{MP4} \textit{Container}).

\smallskip
\textbf{ITU (International Telecommunication Union)}
UN agency that coordinates shared global use of the radio spectrum, satellite orbits and establishes worldwide standards.

\smallskip
\textbf{Keyframe}
A frame of video that contains all of the detail of the scene.

\smallskip
\textbf{LCEVC (Low Complexity Enhancement Video Coding)}
Takes a base video \textit{Codec} such as \textit{AVC} and employs an efficient low-complexity enhancement that correct artifacts produced by the base video codec and adds detail and sharpness for the final output video.

\smallskip
\textbf{LITC (Linear Timecode)}
\textit{SMPTE} \textit{Timecode} data in an audio signal.

\smallskip
\textbf{Luminance (Y)}
The intensity or brightness of a video signal. See also \textit{Chrominance}.

%------------------------------------------------
\section{L-M}
\hrule

\medskip


\smallskip
\textbf{m3u8}
Playlist file.

\smallskip
\textbf{MAM (Media Asset Management)}
Repository for large media files, such as video and full-resolution images. Often called \textit{DAM}'s.

\smallskip
\textbf{Marlin}
Open-standard \textit{DRM}. Widely used in Japan, China, \& East Asia.

\smallskip
\textbf{MBR (Multi-bitrate Streaming)}
Sending out multiple versions of the same live stream at different quality levels.

\smallskip
\textbf{MovieLabs}
Non-profit to R\&D Movie distribution and protection. (Disney, Paramount, Twentieth Century Fox, Sony Pictures, Universal, and Warner Bros).

\smallskip
\textbf{Movielabs Digital Distribution Framework}
Family of complementary, compatible specifications that address key aspects of online delivery e.g. facilitating automation.

\smallskip
\textbf{MP4}
\textit{Container} format.

\smallskip
\textbf{MPEG (Motion Picture Experts Group)}
Standards committee that sets policy for media encoding and decoding standards.

\smallskip
\textbf{MPEG-[1|2|3|4|H]}
See \textit{H.26[1|2|3|4|5]}.

\smallskip
\textbf{MPEG-DASH}
See \textit{DASH}.

\smallskip
\textbf{MSE (Media Source Extensions)}
Spec for JavaScript to send byte streams to media \textit{codecs} within Web browsers.

\smallskip
\textbf{MSO (Multiple System Operator)}
Companies that offer services beyond TV broadcast e.g. Internet and telephone alongside TV.  See \textit{MVPD}.

\smallskip
\textbf{Multicast}
Streaming from one server to many end points simultaneously e.g. \textit{IPTV}.  See also \textit{Unicast}.

\smallskip
\textbf{Multiplex}
Interleaving of multiple signals together into a single signal.

\smallskip
\textbf{Muxing}
See \textit{Multiplex}.

\smallskip
\textbf{MVNO (Mobile Virtual Network Operator)}
Reseller of network services from other telecommunications suppliers and who does not own the telecommunication infrastructure.

\smallskip
\textbf{MVPD (Multichannel Video Programming Distributor)}
Company that provides multiple TV channels e.g. Sky, BT, Comcast, DirecTV. Also known as \textit{MSOs}. See also \textit{vMVPD}.


%------------------------------------------------
\section{N-O}
\hrule

\medskip
\textbf{Narrowcast}
Send data to specific list of recipients.

\smallskip
\textbf{NDI (Network Device Interface)}
Specification to enable video-compatible products to communicate video in frame accurate and suitable for switching in a live production.

\medskip
\textbf{Nielsen Ratings}
Audience measurement system that seeks to determine the audience size of TV programming in the US.

\smallskip
\textbf{NTSC (National Television Standards Committee)}
A television video format used in the United States and elsewhere. Displayed 525 lines of resolution at 60 fields per second, 30 frames per second (actually a fractional value near 29.97). See also \textit{PAL}.

\smallskip
\textbf{OBS (Open Broadcaster Software)}
Free and open source software for video recording and live streaming.

\smallskip
\textbf{Ogg}
An open-source \textit{container} format that primarily houses \textit{Theora} and \textit{Vorbis}.

\smallskip
\textbf{OpenGL}
Cross-language and platform API to render 2D and 3D vector graphics.  Normally used for desktop Apps, written in C. See also \textit{WebGL}.

\smallskip
\textbf{OTT (Over The Top)}
Service offered directly to viewers via the Internet. OTT bypasses cable, broadcast, and satellite television platforms. e.g. Netflix.

\smallskip
\textbf{OVP (Online Video Platform)}
Platform that allows you to stream video content as well as upload your own produced content e.g. Dacast, JW Player, Kaltura.


%------------------------------------------------
\section{P-Q}
\hrule

\medskip
\textbf{PAL (Phase Alternation Line)}
Television video format used in Europe and elsewhere. Displayed with 625 lines of resolution at 50 fields per second, 25 frames per second. See also \textit{NTSC}.

\smallskip
\textbf{Perceptual Compression}
A compression technique that takes advantage of knowledge of how humans perceive; that is, by eliminating visual detail that the eye cannot easily see or audio frequencies that the ear cannot easily hear.

\smallskip
\textbf{PlayReady}
Microsoft \textit{DRM} - Supported on Windows, most STBs and TVs.

\smallskip
\textbf{Post-roll}
Displaying a video advertisement after the primary video is played.

\smallskip
\textbf{Pre-roll}
Displaying a video advertisement before the primary video is played.

\smallskip
\textbf{Progressive Download}
A method of delivering audio/video data over the Internet that involves playing a segment, often the initial downloaded portion of a file, while the rest is downloading.

\smallskip
\textbf{Progressive Video}
Video consisting of complete frames, not interlaced fields. Each individual frame is a coherent image captured by the camera at a single moment in time. See also \textit{Interlaced}.

\smallskip
\textbf{PVOD (Premium VOD)}
Monetisation Model - Variation of \textit{TVOD} where end-users can pay to access content sooner than others.


%------------------------------------------------
\section{R-S}
\hrule

\medskip
\textbf{RGB}
Acronym for Red, Green, Blue. Full-color video signal format, consisting of three elements. See also \textit{YUV}.

\smallskip
\textbf{RTMP (Real-Time Messaging Protocol)}
Protocol for live streaming often use to deliver a stream from encoder to \textit{OVP} endpoint for ingestion.

\smallskip
\textbf{RTP (Real-Time Transport Protocol)}
Network protocol for delivering audio and video over IP networks.

\smallskip
\textbf{RTSP (Real-Time Streaming Protocol)}
Protocol for establishing and controlling media sessions between endpoints, allowing commands such as play, record and pause to control of the media streaming from the server to a client.

\smallskip
\textbf{Segmenting}
See \textit{Fragmentation}.

\smallskip
\textbf{Simulcast (Simultaneous Broadcast)}
A broadcast that is coordinated between two or more mediums, such as radio and television.

\smallskip
\textbf{Smooth Streaming}
Microsoft’s proprietary \textit{ABR} technology. Uses \textit{fMP4} and \textit{PIFF} to stream via HTTP.

\smallskip
\textbf{SMPTE (Society of Motion Picture \& Television Engineers)}
Global professional association of engineers, technologists, and execs working in the M\&E industy.

\smallskip
\textbf{SSAI (Server-Side Ad Insertion)}
See \textit{DAI}.

\smallskip
\textbf{SVOD (Subscription VOD)}
Grants customers access to a catalog of video content for a recurring daily, weekly, or monthly rate.



%------------------------------------------------
\section{T-U}
\hrule

\medskip
\textbf{Theora}
Open-source video \textit{Codec}.

\smallskip
\textbf{Timecode}
An exact time used to identify a specific frame in a clip or production. Measured in hours, minutes, seconds, and frames. Primary types are \textit{SMPTE}, \textit{LITC} and \textit{VITC}.

\smallskip
\textbf{Transcode}
Process of creating copies of video files in different bitrates.

\smallskip
\textbf{Trick Play}
Visual feedback from the player while user is rewinding or fast-forwarding a stream.

\smallskip
\textbf{TS (Video Transport Stream)}
Video data compressed with standard MPEG-2. Often utilised for saving video on a DVD or broadcast video.

\smallskip
\textbf{TVOD (Transactional VOD)}
Monetisation Model - Buying or renting a piece of content.

\smallskip
\textbf{Ultra HD}
Brand name for consumer 4K monitors.

\smallskip
\textbf{UltraViolet}
Defunct cloud-based digital rights locker.

\smallskip
\textbf{Unicast}
Streaming from one server, endpoint, or user. e.g. \textit{OTT}.  See also \textit{Multicast}.


%------------------------------------------------
\section{V}
\hrule

\medskip
\textbf{VAST (Video Ad Serving Template)}
XML schema for serving ads to digital video players, and describes expected video player behavior when executing VAST-formatted ad responses.

\smallskip
\textbf{VBR (Variable Bitrate)}
Allows a higher bitrate (and therefore more storage space) to be allocated to the more complex segments of media files while less space is allocated to less complex segments.

\smallskip
\textbf{VITC (Vertical Interval Timecode)}
\textit{SMPTE} \textit{Timecode} encoded on one scan line in a video signal.

\smallskip
\textbf{VLC (VideoLan Client)}
A player capable of playing a wide variety of formats.

\smallskip
\textbf{vMVPD (Virtual Multichannel Video Programming Distributor)}
Provides multiple TV channels through the internet e.g. DirecTV Now, PlayStation Vue, YouTube TV. See also \textit{MVPD}.

\smallskip
\textbf{VOD (Video on demand)}
Interactive TV system that allows the viewer to select content and view it at a time of their choosing.

\smallskip
\textbf{Vorbis}
Open-source audio \textit{codec}.

\smallskip
\textbf{VP9}
Open and royalty-free video \textit{codec} developed by Google.

\smallskip
\textbf{VTM (VVC Test Model)}
Reference software codebase for testing \textit{VVC}.

\smallskip
\textbf{VTT (Video Text Track)}
Text file in the Web Video Text Tracks (WebVTT) format. Can include subtitles, captions, descriptions, chapters, and metadata.

\smallskip
\textbf{VVC (Versatile Video Coding)}
See \textit{H.266}.


%------------------------------------------------
\section{W-Z}
\hrule

\medskip
\textbf{Watermark}
Embedding data into a video stream or overlaying onto the video for the purpose of later identifying where that video originate. Not to be confused with \textit{Fingerprint}.

\smallskip
\textbf{WebGL}
Designed for rendering 2D and 3D graphhics, running in the browser with Javascript. See also \textit{OpenGL}.

\smallskip
\textbf{WebM}
Codec created as VP8, purchased by Google and open-sourced.

\smallskip
\textbf{WebRTC (Web Real-Time Communication)}
Allows direct peer-to-peer real-time communication via APIs from browser with no plugins.

\smallskip
\textbf{Widevine}
Google \textit{DRM} - Used on Android Devices natively, in Chrome, Edge (soon), Roku, Smart TVs.

\smallskip
\textbf{Windows Media}
Microsoft's proprietary legacy media format that housed audio, video, and metadata.

\smallskip
\textbf{WMV (Windows Media Video)}
Series of video codecs and their corresponding video coding formats developed by Microsoft.

\smallskip
\textbf{YUV}
Full-color video signal format, consisting of three elements: Y (luminance), and U and V (chrominance). See also \textit{RGB}.
