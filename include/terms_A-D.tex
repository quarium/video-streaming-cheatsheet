\section{A-D}
\hrule

\medskip
\textbf{AAC (Advanced Audio Coding)}
Audio \textit{Codec}.

\smallskip
\textbf{ABR (Adaptive Bitrate Streaming)}
Client selects an appropriate \textit{Encode} quality dynamically by analysing the available bandwidth and CPU.

\smallskip
\textbf{ADI (Asset Distribution Interface Specification)}
Means by which content and metadata are transported from a provider to a \textit{MAM}.

\smallskip
\textbf{AIFF (Audio Interchange File Format)}
Audio format developed by Apple.

\smallskip
\textbf{ARPU (Average Revenue Per User)}
Average revenue generated by each user of service.

\smallskip
\textbf{ASF (Advanced System Format)}
A \textit{Container} format defined by Microsoft for media files.

\smallskip
\textbf{ATSC (Advanced Television Systems Committee)}
Not-for-profit, developing voluntary standards for digital television.

\smallskip
\textbf{AV1 (AOMedia Video 1)}
Open, royalty-free \textit{Codec}. Competes with \textit{H.265}.

\smallskip
\textbf{AVI (Audio Video Interleave)}
Proprietary \textit{Container} format by Microsoft.

\smallskip
\textbf{AVC (Advanced Video Coding)}
See \textit{H.264}.

\smallskip
\textbf{AVOD (Advertising VOD)}
Monetisation Model - Advertising supporting \textit{VOD}.

\smallskip
\textbf{Cablelabs (Cable Television Laboratories)}
Not-for-profit innovation and R\&D lab.

\smallskip
\textbf{CBR (Constant Bitrate)}
Encoding each frame or \textit{GOP} at a constant bitrate.

\smallskip
\textbf{CC (Closed Captions)}
Text version of the spoken part of a television, movie.

\smallskip
\textbf{CDN (Content Delivery Network)}
Network allowing content to be physically close to the end-user reducing the latency incurred in retrieving content.

\smallskip
\textbf{CENC (Common Encryption Scheme)}
Standard format for encryption-related metadata for any \textit{DRM}. Videos can be encrypted once and any decryption module can decrypt.  Used with \textit{ISO BMFF}.

\smallskip
\textbf{CMAF (Common Media Application Format)}
\textit{Container} and standards for a single approach to streaming that works with \textit{HLS} and \textit{DASH}.

\smallskip
\textbf{CPM (Cost Per Mile)}
Metric that represents how much money advertisers are spending to show ads.

\smallskip
\textbf{Chrominance}
The colour of a video signal. See also \textit{Luminance (Y)}.

\smallskip
\textbf{Chunking}
See \textit{Fragmentation}.

\smallskip
\textbf{Codec (Compress \& Decompress)}
Technique for reducing the file sizes of video \& audio. e.g. \textit{H.264}.

\smallskip
\textbf{Container}
Wraps into a single file the encoded Audio, Video \& Metadata.  e.g. \textit{MP4} \& \textit{AVI}.

\smallskip
\textbf{D2C (Direct-to-Consumer)}
Business model to deliver content to customers.

\smallskip
\textbf{DAI (Dynamic Ad Insertion)}
Server-side technology that allows the serving of video ads into live linear programming and \textit{VOD} in a seamless TV-like experience without latency or buffering between content and ads.

\smallskip
\textbf{DAM (Digital Asset Managment)}
See \textit{MAM}.

\smallskip
\textbf{DASH (Dynamic Adaptive Streaming via HTTP)}
\textit{ABR} enabling streaming of content via HTTP server. Doesn't require a specific \textit{codec} - could be \textit{H.264}, \textit{VP9}, \textit{H.265}, etc.

\smallskip
\textbf{DASH-IF (DASH Industry Forum)}
Industry group creating interoperability guidelines and promoting \textit{MPEG-DASH}.

\smallskip
\textbf{dash.js}
Reference client implementation by \textit{DASH-IF} for the playback of \textit{MPEG-DASH} via JavaScript and compliant \textit{MSE} / \textit{EME} platforms.

\smallskip
\textbf{DECE (Digital Entertainment Content Ecosystem)}
Inventor of \textit{Ultraviolet}, now dissolved.

\smallskip
\textbf{Deinterlace}
To process interlaced TV video. Uses every other line from one field \& adds new in-between lines. See also \textit{Interlace}.

\smallskip
\textbf{DMCA (Digital Millennium Copyright Act)}
Law that criminalises production and dissemination of technology, devices, or services intended to circumvent measures that control access to copyrighted works.

\smallskip
\textbf{DOCSIS (Data Over Cable Service Interface Specification)}
Telco standard for adding high-bandwidth data transfer to an existing cable television system.

\smallskip
\textbf{DOG (Digital on-screen graphic)}
\textit{Watermark} logo that most television broadcasters overlay over a portion of the screen area of their programs to identify the channel.

\smallskip
\textbf{DRM (Digital Rights Management)}
Technologies that limit the usage \& sharing of digital content.  See \textit{Widevine}, \textit{Playready}, \textit{Fairplay} \& \textit{Marlin}.

\smallskip
\textbf{DVB (Digital Video Broadcasting)}
Open standards for digital TV.  Satalite [DVB-S, DVB-S2], Cable [DVB-SH, DVB-C, DVB-C2], Terrestrial TV [DVB-T, DVB-T2].
